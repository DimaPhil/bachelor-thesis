\documentclass[specification,annotation]{itmo-student-thesis}

%% Опции пакета:
%% - specification - если есть, генерируется задание, иначе не генерируется
%% - annotation - если есть, генерируется аннотация, иначе не генерируется
%% - times - делает все шрифтом Times New Roman, требует пакета pscyr.

%% Делает запятую в формулах более интеллектуальной, например: 
%% $1,5x$ будет читаться как полтора икса, а не один запятая пять иксов. 
%% Однако если написать $1, 5x$, то все будет как прежде.
\usepackage{icomma}

%% Указываем файл с библиографией.
\addbibresource{bachelor-thesis.bib}

\begin{document}

\studygroup{M3439}
\title{Поиск подграфов социального графа, включающих заданное множество пользователей.}
\author{Филиппов Дмитрий Сергеевич}{Филиппов Д.С.}
\supervisor{Фильченков Андрей Александрович}{Фильченков А.А.}{канд. физ.-мат. наук}{доцент кафедры компьютерных технологий}
\publishyear{2017}
%% Дата выдачи задания. Можно не указывать, тогда надо будет заполнить от руки.
\startdate{01}{сентября}{2016}
%% Срок сдачи студентом работы. Можно не указывать, тогда надо будет заполнить от руки.
\finishdate{31}{мая}{2017}
%% Дата защиты. Можно не указывать, тогда надо будет заполнить от руки.
\defencedate{15}{июня}{2017}

\addconsultant{Коршунов А.В.}{канд. физ.-мат. наук, без звания}

%% Задание
%%% Техническое задание и исходные данные к работе
\technicalspec{По данному неориентированному графу социальной сети и выделенным в нем вершинам-запросам требуется найти плотный подграф, содержащий все или почти все данные вершины. Подграф должен иметь небольшой размер, чтобы его было удобно визуализировать.}

%%% Содержание выпускной квалификационной работы (перечень подлежащих разработке вопросов)
\plannedcontents{Пояснительная записка должна демонстрировать новый подход к решению этой задачи, а также его плюсы и минусы по сравнению с предыдущими методами. Должно быть произведено сравнение оптимальности ответов, времени работы, а также других метрик для всех рассмотренных алгоритмов.}

%%% Исходные материалы и пособия 
\plannedsources{\begin{enumerate}
    \item Faloutsos C., Tong H. Center-Piece Subgraphs: Problem Definition and Fast Solutions \cite{Faloutsos06};
    \item N. Ruchansky и др. The Minimum Wiener Connector Problem \cite{Wiener15};
\end{enumerate}}

%%% Календарный план
\addstage{Ознакомление с исходными статьями}{10.2016}
\addstage{Поиск новых статей по теме}{11.2016}
\addstage{Изучение новых статей, выбор бейзлайна}{12.2016}
\addstage{Реализаций бейзлайна}{03.2017}
\addstage{Исследование темы, предложений улучшений бейзлайна}{03.2017}
\addstage{Реализация улучшений}{04.2017}
\addstage{Написание пояснительной записки}{05.2017}

%%% Цель исследования
\researchaim{Научиться более оптимально решать описанную задачу.}

%%% Задачи, решаемые в ВКР
\researchtargets{\begin{enumerate}
    \item Провести исследование описанной задачи;
    \item Реализовать базовое решение;
    \item Придумать метод улучшения базового решения и реализовать его.
\end{enumerate}}

%%% Использование современных пакетов компьютерных программ и технологий
\advancedtechnologyusage{Для реализации алгоритмов был использован язык программирования Java 1.8 с дополнительно установленной библиотекой kryo для работы с индексами.}

%%% Краткая характеристика полученных результатов 
\researchsummary{????}

%%% Гранты, полученные при выполнении работы 
\researchfunding{}

%%% Наличие публикаций и выступлений на конференциях по теме выпускной работы
\researchpublications{}


%% Эта команда генерирует титульный лист и аннотацию.
\maketitle{Бакалавр}

%% Оглавление
\tableofcontents

%% Макрос для введения. Совместим со старым стилевиком.
\startprefacepage

\section{Краткое описание}

Последнее время изучение и аналих социальных сетей представляет большой интерес. 
Один из примеров анализа социальных сетей~--- нахождение сообществ пользователей. 
Многие работы исследуют сообщества целого графа, в то время как большой интерес также представляет анализ сообществ, 
образованных только данным множеством вершин~--- по данным выделенным вершинам в социальном графе найти небольшой плотный подграф, содержащий их все. Эта задача также широко изучена, однако наложение условия на наличие всех вершин в итоговом подграфе не всегда оптимально для нахождения наиболее плотного подграфа, так как этот подход не учитывает возможный <<шум>> в запросе. 

В этой работе по данным графу $G$ и набору вершин $Q$, мы рассматриваем задачу поиска сообщества, содержащего большинство, но не обязательно все вершины из $Q$, используя модель поиска сообщества, основанную на псевдоклике $k$-core. Мы формулируем нашу задачу о нахождении \textit{Оптимального $k$-core сообщества} с максимальным $k$, которое содержит большинство вершин из $Q$. Мы показываем, что эта задача NP-трудная. Поэтому мы разрабатываем эвристический алгоритм, позволяющий найти приблизительное решение этой задачи. Эксперименты на реальных социальных сетях показывают, что наш алгоритм в среднем показывает результаты лучше, чем все предыдущие эксперименты, требовавшие наличие всех вершин в итоговом подграфе.

\startprefacepage

Social networks researching and analysis has become very popular in the century of the social networks. One of the examples of the social networks analysis is a community search problem. Most of the related works explore communities of the whole network whereas the search of community containing only selected vertices of the network also presents a big interest. This task is also researched quite widely, but adding some requirements that make the problem more related to the real life makes it more interesting and harder. We're going to investigate the problem of finding community containing not necessary all the selected vertices, but only most of them, i.e. making the noise in the selected vertices possible.

The relevance of the initial problem, where all selected vertices should be present in the resulting subgraph may be found in many areas:

\begin{enumerate}
  \item Police, security~--- knowing only several suspects, you need to find the whole groupment or gang;
  \item Social networks~--- after adding one or more friends in the social network, you may find suggestions with people that are densely connected with recently added people useful;
  \item Medicine~--- by several infected people you need to find other possibly infected using the social graph of their familiarity;
\end{enumerate}

Our problem (which doesn't require the answer to contain all selected vertices, but only most of them) is relevant in the same spheres, and besides as you can see below, is more relevant to the real life:

\begin{enumerate}
  \item Police~--- find the rest of the groupment, taking into consideration that some of the suspects may be taken only for one deal and doesn't relate to the whole groupment;
  \item Social networks~--- friends recommendation after recent friending, taking into consideration that recently added friends may be from different social communities and may know each other only by you;
  \item Medicine~--- find the most likely infected people, taking into consideration that some of the results may be false-positive;
\end{enumerate}

Almost all existing articles about community search problem are solving the first task, i.e. the community search with all selected vertices in it. We're going to introduce new algorithm which takes query noise into consideration and solves the community search problem better, especially on noisy queries.

In chapter $1$ the main definitions and terms are presented. This chapter also contains a review of the related work and existing solutions. At the end of this chapter we provide updated requirements for our algorithm.

In chapter $2$ our new algorithm is presented. The algorithm is splitted into several phases for better understanding. We also show how our algorithm works on a bunch of examples with illustrations.

In chapter $3$ the experimental part is presented: we describe what datasets that were taken for the experiments, how the test cases and queries were built and how our algorithm works on these experiments comparing to the main baselines.

\chapter{Preliminaries and existing solutions}

\startrelatedwork

Solving different problems like community search in social networks is very relevant in the current world. There are different methods for finding communities in the whole network \cite{Newman04, Newman06, Fortunato10, Cui13} and also for finding a dense community containing all selected vertices in the network \cite{Faloutsos06, Wiener15, Huang15, Barbieri15}. There are even algorithms for splitting selected vertices into several communities \cite{Akoglu13, Bian18}. However, almost all of these algorithms don't support noise in the query and find non-optimal subgraphs on such queries. We are going to solve this problem in our work, suggesting the new algorithm that will effectively find the needed subgraph even if there is some noise in the query.

\section{Terms and definitions}

\subsection{Graph terms}

In this section we will describe all terms and definitions that may be helpful for the further reading.

Let's define \textbf{$N_G(v)$} as the set of the neighbors of vertex $v$ in graph $G$, i.e. the set of the vertices that are directly connected to $v$ by edge: $N_G(v) = \{u | (v, u) \in E(G)\}$. If graph $G$ can be obviously recognized from the context, we can use just $N(v)$.

\textbf{$G[V]$} is called \textbf{originated subgraph} of the graph $G$ by the set of vertices $V$ if $G[V] := (V, E[G, V])$, where $E[G, V]$~--- the subset of the set of edges of $G$, both ends of which are contained in $V$, i.e. $E[G, V] = E(G) \cap (V \times V)$.

\textbf{k-truss} of the graph $G$ is the subgraph $G' \subseteq G$ containing the maximal possible number of vertices, such that for each edge $(v, u)$ the number of vertices $w$, such that edges $(w, v)$ and $(w, u)$ exists in $G'$ is at least $k$. In other words, $k-truss$ is the maximal by size subgraph $G'$, for each of which edge $(v, u)$, the $|N_{G'}(v) \cap N_{G'}(u)| >= k$ is true.

\textbf{k-core} of the graph $G$ is called the maximal by the number of vertices subgraph $G' \subseteq G$, so that the degree of each of its vertices is at least $k$. For the fixed $k$, by $C_k$ we will denote \textit{$k$-core}, namely the set of the connected components of which it consists. So, $C_k = \{H_i\}$, where $H_i$ is the $i$-th connected component, where the degree of each vertex is at least $k$. The number $k$ we will call the \textbf{order} of \textit{k-core}.

By \boldmath$\mu(G)$\unboldmath we will denote the minimal degree of the vertices $G$, i.e. $\mu(G) = \min_{v \in V(G)} deg(v)$.

\textbf{Core decomposition} is the set of \textit{k-core} for all possible $k$: $C = \{C_k\}_{k=1}^{k=k^*}$. We also need to clarify that from the definition of the \textit{k-core} you can see that $C_1 \supseteq C_2 \supseteq C_3 \ldots \supseteq C_{k^*}$ (where $k^*$ is the maximal possible $k$ in core decomposition).

\textbf{Сore index} for the vertex $v$ is called the minimal by the size \textit{k-core} which includes $v$, i.e. \textit{k-core} with the maximal $k$: $c(v) = \max(k \in [0..k^*] | v \in C_k)$.

\textbf{$\gamma$-quasi-clique} of the graph $G$ is called any such subgraph $G' \subseteq G$ that it is <<dense enough>>, i.e. $\frac{2 \cdot |E(G')|}{|V(G')| \cdot (|V(G')| - 1)} \ge \gamma$.

\subsection{Social networks}

\textbf{The community} or \textbf{The community in social network} is called the set of vertices of the social network $G$, where all vertices are united by some property or attribute. For example, <<the community of rock lovers>> or <<the community of Apple shareholders>>.

\textbf{The social clique} or \textbf{clique} we will call the set of people in social network, where everyone "knows" (i.e. is connected by edge) each other, in other words when between any pair of distinct people there is an edge in social network.

\textbf{Social pseudoclique} or \textbf{pseudoclique} we will call the set of people, where it is not required that each pair of distinct people is connected by edge, but this set is still densely connected. The estimation, how dense the pseudoclique is connected depends on the type of the pseudoclique and will be discussed later in the work, but in all definitions the biggest role plays the number of edges in subgraph in comparison with the number of pairs of vertices ($\frac{2 \cdot |E(G)|}{|V(G)| \cdot (|V(G)| - 1)}$).

\subsection{Useful abbreviations}

\textbf{CSP}~--- Community Search Problem. This is the problem for finding the community in the social network which contains all the selected vertices.

\textbf{NCSP}~--- Noising Community Search Problem. This is the problem for finding the community in the social network which contains most of the selected vertices, but not necessary all (not including the noise).

\section{Overview}

The problem that we are analyzing in the work is formulated as follows: given an undirected unweighted graph $G$ and a set of selected vertices $Q \subset V(G)$, the goal is to solve noising community search problem (NCSP)~--- to find the community which contains most of the vertices from $Q$, but not necessary all of them. Sometimes we will call vertices from $Q$ <<query vertices>>, <<query>>, or <<vertices from query>>.

\subsection{Related work}

The most part of algorithms for solving SCP are the algorithms based on finding the optimal pseudocliques with some additional heuristics. There are a lot of different pseudocliques that were considered in different articles: \textit{k-core} \cite{Barbieri15}, \textit{k-truss} \cite{Huang15}, \textit{$\gamma$-quasi-clique} \cite{Zhu11} or just algorithms that maximize the edge density in the resulting subgraph \cite{Wu15} which is almost a definition of a pseudoclique. For each of these pseudocliques the algorithms are evolving and becoming better, optimizing the previous results using new heuristics. Comparing the results of the algorithms that use different pseudocliques is quite hard and unlikely will give visible results because of the difference of the metrics that are being optimized~--- the result strongly depends on the initial graph and the queries on it. In some cases one pseudoclique will obtain results better than others, but in other cases it will work worse, so actually it's worth to compare some common performance metrics, but unfortunately it doesn't give us the whole understanding of the optimality or non-optimality of the algorithms.

\section{Final requirements for our work}

Most of the current solutions solve CSP quite optimal~--- each of the solutions uses it's own metric and obtains quite good results. However, solutions for NCSP (which includes noise into consideration) are quite rare, despite of this problem is more related to the real life. We found two articles that are able to solve NSCP at least somehow: C. Faloutsos \& H. Tong \cite{Faloutsos06} and A. Gionis et al. \cite{Gionis15}, however the last problem is not focused on solving NCSP (but solves it at the same time). So, the goal of our article is to build the algorithm that focuses on NCSP solving and obtains better results than the current ones. Here are some requirements for our algorithm:

\begin{itemize}
    \item The algorithm should obtain better results than the current ones \cite{Faloutsos06, Gionis15, Barbieri15};
    \item The algorithm should be quite optimal, ideally not loosing the competition with other algorithms in terms of working time;
    \item It would be an advantage to support backwards compatibility~--- if the user wants to find subgraph that contains \textit{all} query vertices, it should be possible to be done.
\end{itemize}

\input{chapters/conclusion.tex}

\printmainbibliography

\end{document}
