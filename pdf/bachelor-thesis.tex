\documentclass[specification,annotation]{itmo-student-thesis}

%% Опции пакета:
%% - specification - если есть, генерируется задание, иначе не генерируется
%% - annotation - если есть, генерируется аннотация, иначе не генерируется
%% - times - делает все шрифтом Times New Roman, требует пакета pscyr.

%% Делает запятую в формулах более интеллектуальной, например: 
%% $1,5x$ будет читаться как полтора икса, а не один запятая пять иксов. 
%% Однако если написать $1, 5x$, то все будет как прежде.
\usepackage{icomma}

%% Указываем файл с библиографией.
\addbibresource{bachelor-thesis.bib}

\begin{document}

\studygroup{M3439}
\title{Поиск подграфов социального графа, включающих заданное множество пользователей.}
\author{Филиппов Дмитрий Сергеевич}{Филиппов Д.С.}
\supervisor{Фильченков Андрей Александрович}{Фильченков А.А.}{канд. физ.-мат. наук}{доцент кафедры компьютерных технологий}
\publishyear{2017}
%% Дата выдачи задания. Можно не указывать, тогда надо будет заполнить от руки.
\startdate{01}{сентября}{2016}
%% Срок сдачи студентом работы. Можно не указывать, тогда надо будет заполнить от руки.
\finishdate{31}{мая}{2017}
%% Дата защиты. Можно не указывать, тогда надо будет заполнить от руки.
\defencedate{15}{июня}{2017}

\addconsultant{Коршунов А.В.}{канд. физ.-мат. наук, без звания}

%% Задание
%%% Техническое задание и исходные данные к работе
\technicalspec{По данному неориентированному невзвешенному графу социальной сети и выделенным в нем вершинам-запросам требуется найти плотный подграф, содержащий все или большинство данных вершин, то есть так называемое сообщество социальной сети.}

%%% Содержание выпускной квалификационной работы (перечень подлежащих разработке вопросов)
\plannedcontents{Пояснительная записка должна демонстрировать новый подход к решению этой задачи, а также его плюсы и минусы по сравнению с предыдущими методами. Должно быть произведено сравнение оптимальности ответов, времени работы, а также других метрик для всех рассмотренных и приведенных алгоритмов.}

%%% Исходные материалы и пособия 
\plannedsources{\begin{enumerate}
    \item Faloutsos C., McCurley K.S. and Tomkins A. Fast discovery of connection subgraphs;
    \item Faloutsos C., Tong H. Center-Piece Subgraphs: Problem Definition and Fast Solutions;
    \item Ruchansky N. et al. The Minimum Wiener Connector Problem;
    \item Gionis A., Mathioudakis M. and Ukkonen A. Bump hunting in the dark: Local discrepancy maximization on graphs.
\end{enumerate}}

%%% Календарный план
\addstage{Ознакомление с исходными статьями}{10.2016}
\addstage{Поиск новых статей}{11.2016}
\addstage{Изучение новых статей, выбор бейзлайна}{12.2016}
\addstage{Реализация бейзлайна}{03.2017}
\addstage{Исследование темы, предложения улучшений бейзлайна}{03.2017}
\addstage{Реализация улучшений, сравнение практических результатов с теоретическими}{04.2017}
\addstage{Написание пояснительной записки}{05.2017}

%%% Цель исследования
\researchaim{В рамках работы необходить предложить улучшение существующих методов поиска сообщества по выделенным в графе социальной сети вершинам, предложить алгоритм отсеивания шума в запросе для получения более плотного подграфа в качестве ответа.}

%%% Задачи, решаемые в ВКР
\researchtargets{\begin{enumerate}
    \item Провести исследование описанной задачи;
    \item Выделить одно или несколько базовых решений;
    \item Реализовать выбранные базовые решения и сравнить их результаты с результатами с статьях;
    \item Разработать методы улучшения базовых решений, способные учитывать шум в запросах;
    \item Реализовать разработанные методы и сравнить полученные результаты с результатами базовых решений.
\end{enumerate}}

%%% Использование современных пакетов компьютерных программ и технологий
\advancedtechnologyusage{Для реализации алгоритмов был использован язык программирования \textit{Java 1.8}. Также был использован фреймворк \textit{Kryo} для быстрой и автоматической сериализации и десериализации большого объема данных. Для вычислений, использующих большой объем памяти, были использованы технологии \textit{ssh} и \textit{slurm} для подключения и работы на серверах кластера Универстита ИТМО, имеющих $128$, $256$ и $496$ Гб оперативной памяти.}

%%% Краткая характеристика полученных результатов 
\researchsummary{Результаты, полученные в статье, показывают, что на запросах, содержащих шум, предложенное решение работает оптимальнее всех предыдущих. На запросах без шума решение работает не хуже, а иногда даже лучше существующих решений.}

%%% Гранты, полученные при выполнении работы 
\researchfunding{}

%%% Наличие публикаций и выступлений на конференциях по теме выпускной работы
\researchpublications{}


%% Эта команда генерирует титульный лист и аннотацию.
\maketitle{Бакалавр}

%% Оглавление
\tableofcontents

%% Макрос для введения. Совместим со старым стилевиком.
\startprefacepage

\section{Краткое описание}

Последнее время изучение и аналих социальных сетей представляет большой интерес. 
Один из примеров анализа социальных сетей~--- нахождение сообществ пользователей. 
Многие работы исследуют сообщества целого графа, в то время как большой интерес также представляет анализ сообществ, 
образованных только данным множеством вершин~--- по данным выделенным вершинам в социальном графе найти небольшой плотный подграф, содержащий их все. Эта задача также широко изучена, однако наложение условия на наличие всех вершин в итоговом подграфе не всегда оптимально для нахождения наиболее плотного подграфа, так как этот подход не учитывает возможный <<шум>> в запросе. 

В этой работе по данным графу $G$ и набору вершин $Q$, мы рассматриваем задачу поиска сообщества, содержащего большинство, но не обязательно все вершины из $Q$, используя модель поиска сообщества, основанную на псевдоклике $k$-core. Мы формулируем нашу задачу о нахождении \textit{Оптимального $k$-core сообщества} с максимальным $k$, которое содержит большинство вершин из $Q$. Мы показываем, что эта задача NP-трудная. Поэтому мы разрабатываем эвристический алгоритм, позволяющий найти приблизительное решение этой задачи. Эксперименты на реальных социальных сетях показывают, что наш алгоритм в среднем показывает результаты лучше, чем все предыдущие эксперименты, требовавшие наличие всех вершин в итоговом подграфе.

\startprefacepage

Social networks researching and analysis has become very popular in the century of the social networks. One of the examples of the social networks analysis is a community search problem. Most of the related works explore communities of the whole network whereas the search of community containing only selected vertices of the network also presents a big interest. This task is also researched quite widely, but adding some requirements that make the problem more related to the real life makes it more interesting and harder. We're going to investigate the problem of finding community containing not necessary all the selected vertices, but only most of them, i.e. making the noise in the selected vertices possible.

The relevance of the initial problem, where all selected vertices should be present in the resulting subgraph may be found in many areas:

\begin{enumerate}
  \item Police, security~--- knowing only several suspects, you need to find the whole groupment or gang;
  \item Social networks~--- after adding one or more friends in the social network, you may find suggestions with people that are densely connected with recently added people useful;
  \item Medicine~--- by several infected people you need to find other possibly infected using the social graph of their familiarity;
\end{enumerate}

Our problem (which doesn't require the answer to contain all selected vertices, but only most of them) is relevant in the same spheres, and besides as you can see below, is more relevant to the real life:

\begin{enumerate}
  \item Police~--- find the rest of the groupment, taking into consideration that some of the suspects may be taken only for one deal and doesn't relate to the whole groupment;
  \item Social networks~--- friends recommendation after recent friending, taking into consideration that recently added friends may be from different social communities and may know each other only by you;
  \item Medicine~--- find the most likely infected people, taking into consideration that some of the results may be false-positive;
\end{enumerate}

Almost all existing articles about community search problem are solving the first task, i.e. the community search with all selected vertices in it. We're going to introduce new algorithm which takes query noise into consideration and solves the community search problem better, especially on noisy queries.

In chapter $1$ the main definitions and terms are presented. This chapter also contains a review of the related work and existing solutions. At the end of this chapter we provide updated requirements for our algorithm.

In chapter $2$ our new algorithm is presented. The algorithm is splitted into several phases for better understanding. We also show how our algorithm works on a bunch of examples with illustrations.

In chapter $3$ the experimental part is presented: we describe what datasets that were taken for the experiments, how the test cases and queries were built and how our algorithm works on these experiments comparing to the main baselines.

%% Начало содержательной части.
\chapter{Первая глава}

\chapterconclusion

В конце каждой главы желательно делать выводы. Вывод по данной главе~--- нумерация работает корректно, ура!

%% Макрос для заключения. Совместим со старым стилевиком.
\startconclusionpage

By given graph $G$ and a set of query vertices $Q \subset V(G)$ in this article we're solving the task of finding the community containing all or most of the vertices from $Q$. As we proved before, the task is very actual nowdays and can be applied in many fields.

In this work we try to find \textit{k-core} with maximal $k$ of the minimal size containing most of the vertices from $Q$. We proof that the task is NP-hard and describe some heuristics for its solution. Experiment results held on the real data show us that on the queries without any noise our suggested solutions works almost the same as the best existing solution. However, after adding a noise to the query, solutions of other authors start to find non-optimal big subgraph when our solution finds densely connected small subgraphs.

Also, the suggested algorithm is backward compatible~--- we can add parameter $minSize$ which will be equal to the minimal number of query vertices in the answer. This parameter won't affect much on our solution, but will help us to find subgraph containing all query vertices.

In future we plan to spread the suggested idea on weighted graph and maybe some other kind of graphs~--- attibuted graphs, multigraphs, etc.


\printmainbibliography

\end{document}
