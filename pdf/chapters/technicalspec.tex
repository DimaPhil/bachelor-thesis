%% Задание
%%% Техническое задание и исходные данные к работе
\technicalspec{По данному неориентированному графу социальной сети и выделенным в нем вершинам-запросам требуется найти плотный подграф, содержащий все или почти все данные вершины. Подграф должен иметь небольшой размер, чтобы его было удобно визуализировать.}

%%% Содержание выпускной квалификационной работы (перечень подлежащих разработке вопросов)
\plannedcontents{Пояснительная записка должна демонстрировать новый подход к решению этой задачи, а также его плюсы и минусы по сравнению с предыдущими методами. Должно быть произведено сравнение оптимальности ответов, времени работы, а также других метрик для всех рассмотренных алгоритмов.}

%%% Исходные материалы и пособия 
\plannedsources{\begin{enumerate}
    \item Faloutsos C., Tong H. Center-Piece Subgraphs: Problem Definition and Fast Solutions \cite{Faloutsos06};
    \item N. Ruchansky и др. The Minimum Wiener Connector Problem \cite{Wiener15};
\end{enumerate}}

%%% Календарный план
\addstage{Ознакомление с исходными статьями}{10.2016}
\addstage{Поиск новых статей по теме}{11.2016}
\addstage{Изучение новых статей, выбор бейзлайна}{12.2016}
\addstage{Реализаций бейзлайна}{03.2017}
\addstage{Исследование темы, предложений улучшений бейзлайна}{03.2017}
\addstage{Реализация улучшений}{04.2017}
\addstage{Написание пояснительной записки}{05.2017}

%%% Цель исследования
\researchaim{Научиться более оптимально решать описанную задачу.}

%%% Задачи, решаемые в ВКР
\researchtargets{\begin{enumerate}
    \item Провести исследование описанной задачи;
    \item Реализовать базовое решение;
    \item Придумать метод улучшения базового решения и реализовать его.
\end{enumerate}}

%%% Использование современных пакетов компьютерных программ и технологий
\advancedtechnologyusage{Для реализации алгоритмов был использован язык программирования Java 1.8 с дополнительно установленной библиотекой kryo для работы с индексами.}

%%% Краткая характеристика полученных результатов 
\researchsummary{????}

%%% Гранты, полученные при выполнении работы 
\researchfunding{}

%%% Наличие публикаций и выступлений на конференциях по теме выпускной работы
\researchpublications{}
