\section{Актуальность}

Актуальность исходной задачи (требующей наличие всех выделенных вершин в итоговом подграфе) проявляется во многих областях:
\begin{enumerate}
  \item Медицина~--- по нескольким заболевшим определить других наиболее вероятно инфицированных, используя социальный граф связей и знакомств;
  \item Полиция~--- зная нескольких подозреваемых, выяснить, кто еще мог быть соучастником преступления, участником банды или группировки;
  \item Социальные сети~--- после добавления одного или нескольких друзей в социальной сети, также предлагается еще несколько, которые тесно связаны с недавно добавленными и которых вы вероятно всего тоже знаете;
  \item Организация мероприятий~--- если на важное мероприятие требуется позвать несколько спикеров, также понять, кого еще, тесно связанного с этими людьми, хорошо было бы увидеть на этом мероприятии.
\end{enumerate}

Актуальность нашей задачи (не требующей наличие всех, а только большинства выделенных вершин в итоговом подграфе) также проявлется в этих областях:

\begin{enumerate}
  \item Медицина~--- определить наиболее вероятно инфицированных, учитывая тот фактор, что некоторые результаты могли оказаться ложно-положительными;
  \item Полиция~--- определить остальных соучастников преступления, участников банды или группировки, учитывая, что некоторые подозреваемые могли быть взяты только для этого дела и к группировке отношения не имеют;
  \item Социальные сети~--- рекомендация друзей после недавнего добавления нескольких людей, учитывая то, что добавленные друзья могут быть из разных социальных сообществ и связь между ними может быть только через вас;
  \item Организация мероприятий~--- приглашение людей, наиболее связанных со спикерами, учитывая то, что спикеры могут быть слабо связаны друг с другом.
\end{enumerate}

%\section{Цель}

%Цель этой бакалаврской работы~--- изучить поставленную задачу и ее текущие методы решения, предложить более эффективный метод решения задачи.

\section{Новизна}

Все найденные по этой теме статьи рассматривают задачу поиска подграфа, содержащего все выделенные вершины. Наш же алгоритм вводит возможность взятия в итоговое подмножество не всех выделенных вершин, а только большинства из них и в среднем показывает результаты лучше, чем предыдущие алгоритмы.

\section{Структура работы}

В главе $N$ рассмотрен вопрос......
