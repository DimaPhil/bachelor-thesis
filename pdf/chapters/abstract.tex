%% Макрос для введения. Совместим со старым стилевиком.
\startprefacepage

\section{Краткое описание}

Последнее время изучение и аналих социальных сетей представляет большой интерес. 
Один из примеров анализа социальных сетей~--- нахождение сообществ пользователей. 
Многие работы исследуют сообщества целого графа, в то время как большой интерес также представляет анализ сообществ, 
образованных только данным множеством вершин~--- по данным выделенным вершинам в социальном графе найти небольшой плотный подграф, содержащий их все. Эта задача также широко изучена, однако наложение условия на наличие всех вершин в итоговом подграфе не всегда оптимально для нахождения наиболее плотного подграфа, так как этот подход не учитывает возможный <<шум>> в запросе. 

В этой работе по данным графу $G$ и набору вершин $Q$, мы рассматриваем задачу поиска сообщества, содержащего большинство, но не обязательно все вершины из $Q$, используя модель поиска сообщества, основанную на псевдоклике $k$-core. Мы формулируем нашу задачу о нахождении \textit{Оптимального $k$-core сообщества} с максимальным $k$, которое содержит большинство вершин из $Q$. Мы показываем, что эта задача NP-трудная. Поэтому мы разрабатываем эвристический алгоритм, позволяющий найти приблизительное решение этой задачи. Эксперименты на реальных социальных сетях показывают, что наш алгоритм в среднем показывает результаты лучше, чем все предыдущие эксперименты, требовавшие наличие всех вершин в итоговом подграфе.
